\chapter{Introduction}
\label{ch:intro}

% \begin{goals}
%   \begin{enumerate}
%   \item
%     \textbf{Review} the module syllabus.
%   \item
%     \textbf{Describe} the need for operating systems.
%   \item
%     \textbf{Define} at a high level the key jobs of an operating system.
%   \end{enumerate}
% \end{goals}

\section{Module introduction}
\label{sec:module-introduction}

\section{Role of operating system}
\label{sec:role-of-operating-system}

Operating systems are the \textbf{glue} between application software and
the hardware it uses.

\subsection{Hardware}
\label{sec:hardware}

\begin{itemize}

\item
  Central Processing Unit (CPU)
\item
  Random Access Memory (RAM)
\item
  Long-term storage devices (Hard disk, SSD)
\item
  I/O devices
\end{itemize}


\subsection{Software}
\label{sec:software}

\begin{itemize}

\item
  Program
\item
  Process
\item
  Thread
\end{itemize}


\section{Why study operating systems?}
\label{sec:why-study-operating-systems}

\begin{itemize}

\item
  \textbf{Missing link} between the hardware and programming concepts
  like JavaScript that are readily understandable.
\item
  \textbf{Concurrent programming} becomes a lot clearer if you know
  what's controlling the concurrent program's activity.
\item
  \textbf{Unusual hardware} will make more sense to you as a programmer.
  Not every program runs on a PC or smartphone!
\end{itemize}

\section{Advice}
\label{sec:advice}

\begin{itemize}
\item
  Lectures and labs will follow a parallel path in places.
  Some concepts will be presented only in the lab sessions, others only in the lecture, others again both.
  \textbf{Don't rely on the notes alone!}
\item
  Will be some \emph{investigative work} and \emph{background reading} on your part.
\item
  Worth considering what you want from the module vis-a-vis your onward career.
\end{itemize}

