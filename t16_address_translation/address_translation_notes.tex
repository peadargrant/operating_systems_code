\chapter{Address translation}
\label{ch:address-translation}

% \begin{goals}
%   \begin{enumerate}
%   \item \textbf{Describe} the mechanism of address translation to relocate a process' address space.
%   \item \textbf{Specify} necessary hardware and OS support for address translation.
%   \end{enumerate}
% \end{goals}

% \begin{reading}
%   \begin{enumerate}
%   \item
%     \citet[Chapter 15]{arpaci-dusseau:2015:operating}\\
%     \url{http://pages.cs.wisc.edu/~remzi/OSTEP/vm-mechanism.pdf}
%   \end{enumerate}
% \end{reading}

\section{Goal}

We want to map a virtual per-process address space $V$ in to the physical shared address space $P$.

\section{Nomenclature}

Assume that we are considering some process $x$: 

\begin{itemize}
\item Machine has address space from $0$ to $P_1$.
\item $x$ occupies physical address space from $P^x_0$ to $P^{x}_1$.
\item $V$ is a virtual address from $0$ to $V^x_1$ as seen by $x$.
\end{itemize}

\section{Assumptions}

To develop the key ideas, we make the following assumptions about the address space of each process in the system:

\begin{enumerate}
\item continuguous in physical memory.
\item less than total size of physical memory.
\item same size as all other process' address spaces.
\end{enumerate}

We will also assume that a process is loaded from its processor-specific machine language representation (i.e. is not compiled or interpreted at runtime!)

\section{Relocation by offset}

Relocation is merely dividing up each memory space to provide an offset.
Firstly we assume that a virtual address $Vx$ that process $x$ wishes to access is translated to the physical address $P$:
\begin{align}
  P & = f( V, x )
\end{align}

The simplest type of relocation we can have involves an offset:
\begin{align}
  P & = V + P_0^x \label{eq:offset}
\end{align}

This idea can be implemented statically or dynamically.

\section{Static relocation}

Static relocation involves the use of a \textbf{loader} that rewrites all memory addresses in the code to implement \autoref{eq:offset}.
Although simple, this idea is no longer used in practice: 

\begin{itemize} 
\item No inter-process bounds-checking.
\item Cannot move process' memory once loaded (see later on).
\end{itemize}

\section{Dynamic relocation}

Dynamic relocation relies on hardware to support to provide bounds-checked offsets in hardware.
First described by \citet{saltzer:1974:the-protection}, it relies on two registers in the CPU that defines the chunk of physical memory assigned to a process.

\begin{description}
\item[Base register] specifies the physical address where the process' address space starts (i.e. the offset).
\item[Bounds register] specifies the physical address where the process' address space ends.
\end{description}

\subsection{Dynamic relocation process}

At load time, OS sets base register to the location in memory where the address space starts.

When the program accesses address $V$:
\begin{enumerate}
\item $P=V+P_0^x$ used to generate $P$.
\item If $P>P^x_1$, exception raised.
\item Otherwise, access transparently happens.
\end{enumerate}

Although simple, these operations rely on both hardware and OS support.

\subsection{Out-of-bounds}

If $P>P^x_1$ an exception is raised.  Control returns to the OS, which will normally terminate the offending process.

\section{Hardware support}

\citet[Figure 15.3]{arpaci-dusseau:2015:operating} summarises the hardware features required to implement dynamica relocation. 
Briefly, CPU must provide:
\begin{description}
\item[Registers (x2)] to store the base and bounds addresses.
  \begin{itemize}
  \item Must not be accessible in User mode! So, CPU requires user and kernel modes (assumed!)
  \item Special instructions in instruction set to update these.
  \end{itemize}
\item[Arithmetic] circuitry to calculate offsets, \autoref{eq:offset}.
\item[Logic] circuitry to detect out-of-bounds condition.
\item[Exception] raising and handling abilities.
\end{description}
These registers and their accompanying logic form part of the Memory Management Unit (MMU) component of the CPU.


\section{OS support}

The operating system must support the hardware in a few key areas:

\subsection{Free List}

OS must keep track of what processes are using what space. This is normally done in a data structure named a ``free list'' by the operating system: 
\begin{enumerate}
\item Identify space when loading a process using the free list.
\item Update the free list on loading a process.
\item Update the free list when a process terminates / is terminated.
\end{enumerate}

\subsection{Context switch}

When a context switch occurs, the base and bounds pointers must be saved and restored along with the rest of the machine state.

These must be kept in a per-process data structure by the operating system.

\subsection{Exception handlers}

The OS must provide the appropriate exception handlers and set them up at boot time.
CPU instructions necessary to do this are usually privilaged.

\section{Problem}

What about the unused space?


%%% Local Variables:
%%% mode: latex
%%% TeX-master: "os_notes"
%%% End:

