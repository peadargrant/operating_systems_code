\chapter{GitLab setup}
\label{ch:gitlab-setup}

For the in-lab assessment you will record your weekly work in a GitLab repository.
Your repository \textbf{MUST} be called \texttt{os\_labs}. 
To assess your weekly work you will add the lecturer read-only to your repository.
The steps below walk you through the process. 

\textbf{It is very important that you follow the required file naming structures!}

\section{Check if the Git command is installed}

The lab machines already have Git installed.

Macs should have Git installed but may need to install XCode to make the command available.

Git is available for windows at:
\url{https://git-scm.com/download/win}


\section{Create your GitLab repository}

\begin{enumerate}
\item Login to \url{https://gitlab.comp.dkit.ie} with your DkIT username / password.
\item Hit the blue \textbf{New Project} button.
\item Choose \textbf{Create Blank Project}.
\item For \textbf{Project name} enter \textit{exactly} \texttt{os\_labs}. \label{step:gitlab-setup-project-name}
\item Drop down the \textbf{Pick a Group or Namespace} and in the \textit{Users} section pick your own username.
\item The project slug field will auto-fill the project name you entered in step~\ref{step:gitlab-setup-project-name}.
\item Leave \textbf{Visibility level} set to private.
\item Make sure \textbf{Initialize repository with a README} is turned on.
\item Click \textbf{Create Project}.
\end{enumerate}

\section{Work with your Git repository}

To work with your GitLab repository we need to clone it to a local Git repository on your computer.
We will then modify some files, add them, commit them and push the changes back to your GitLab repository. 

\begin{enumerate}
\item In GitLab find the front page of your \texttt{os\_labs} project and click the blue \texttt{Clone} button.
\item Under \textbf{Clone with HTTPS} click the Copy button beside the link. 
\item Use PowerShell / Terminal (on Mac) to navigate to the folder you want to keep your \texttt{os_labs} folder in.
\item Type \texttt{git clone} and a space and then paste (right click) your link, e.g.
\begin{verbatim}
git clone https://gitlab.comp.dkit.ie/grantp/os_labs.git
\end{verbatim}
\item Follow the instructions to authenticate to GitLab.
\item Change directory into your \texttt{os\_labs} folder:
\begin{verbatim}
cd os_labs
\end{verbatim}
\item Make a folder \texttt{week01} by typing:
\begin{verbatim}
mkdir week01
\end{verbatim}
\item Use Notepad++ (or any other editor) to modify the README.md file to simply include your name.
\item Add a file in week01 named \textit{exactly} \texttt{firstlab.txt} stating that you have setup your Git repository.
\item Add both of your newly created files to the \textit{staging area} in Git:
\begin{verbatim}
git add README.md
git add week01/firstlab.txt
\end{verbatim}
\item Confirm both have been added by typing
\begin{verbatim}
git status
\end{verbatim}
\item Commit (or snapshot) the change by typing:
\begin{verbatim}
git commit -m 'first lab'
#          ^
#          |
#          the -m switch means next argument is the commit message
\end{verbatim}
\item Push the changes to GitLab using
\begin{verbatim}
git push
\end{verbatim}
\item Confirm on your web browser you can see the changes in GitLab.

\end{enumerate}

\section{Adding lecturer}

\begin{enumerate}
\item Open your \texttt{os\_labs} project in GitLab if not already open. 
\item Under \textbf{Manage} click \textbf{Members}.
\item Click \textbf{Invite Members}.
\item In \textbf{Username} enter \texttt{grantp}.
\item Under \textbf{Role} choose \textit{Guest}.
\item Click \textbf{Invite}.
\end{enumerate}