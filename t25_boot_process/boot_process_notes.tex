\chapter{Boot process}
\label{ch:boot-process}

\section{Boot process}

\begin{enumerate}
\item ROM BIOS loads first-stage bootloader from hard disk.
\item First-stage bootloader runs from RAM and loads second-stage boot-loader from hard disk.
\item Second-stage bootloader loads OS into RAM.
\item Control is transferred to OS.
\item Boot process is then complete.
\end{enumerate}

\section{Network booting}

Normally the operating system is loaded from Hard disk.
It can be instead booted directly from the network (even without a hard disk).

\begin{enumerate}
\item ROM loads PXE boot code.
\item PXE issues DHCP request.
\item DHCP response includes a \texttt{next-server} option that provides IP address of server to load boot file from.
\item Boot file loaded using Trivial File Transfer Protocol (TFTP).
\item Control is transferred to the OS. 
\end{enumerate}

The OS needs to be able to operate in the absence of a fixed hard disk.
Most currently can't by default.
Linux and Windows Embedded can however be configured to operate from a RAM disk or NAS-provided root folder.

PXE is often used however to load the minimal OS environments for Windows and Linux installers to run. 


\section{Dual-boot systems}

Dual-boot systems were more commonly employed by technical users needing to utilise multiple different OS environments.
Often this was Linux and Microsoft Windows on PC-based systems, and OSX and Windows on Mac.
This has decreased in recent years as virtualisation has supplanted it.

The first-stage bootloader loads a second-stage boot loader such as GRand Unified Bootload (GRUB).
This is configurable such that different partitions holding an OS can be booted or chain-loaded.

