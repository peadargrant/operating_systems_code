\chapter{GDB memory lab}
\label{ch:gdb-memory-lab}

\section{Introduction}

Today's lab will look at some basic examination of memory using GDB.
Our programs today are written in C.

The GNU DebuGger (GDB) is a program that lets you run programs line by line, inspecting variables and memory as you go.
See \href{https://www.sourceware.org/gdb/}{Full capability list}.
Debuggers in general are very useful tools regardless of what language you're working in.
We will be using it to get a very basic understanding of how process memory is laid out.


\section{Setup}

\begin{enumerate}
\item Log into your Linux VM on XOA.
\item Update the \texttt{operating\_systems\_code} git repository using:
\begin{verbatim}
cd operating_systems_code
git pull --rebase
\end{verbatim}
\item Go back to your home folder using
\begin{verbatim}
cd
\end{verbatim}
\item In your \texttt{os\_labs} folder create a \texttt{week06} subdirectory and CD into it.
\begin{verbatim}
cd os_labs
mkdir week06
cd week06
\end{verbatim}
\item Copy the files for today's lab using and confirm using \texttt{ls} that they're in your \texttt{week06} folder:
\begin{verbatim}
cp ../../operating_systems_code/*_gdb_memory_lab/* .
ls -l
\end{verbatim}
\end{enumerate}

\section{Memory space exploration}

\begin{enumerate}

\item
  Use the \texttt{cat} command to print the \texttt{program\_monolith} program to the screen.

\item
  Read through the program and confirm that you understand what each line principally does.
  (Don't worry that you are not familiar with C.)

\item
  Compile the program using \texttt{gcc} as follows:
\begin{verbatim}
gcc -g -o program_monolith.exe program_monolith.c
\end{verbatim}
  The \texttt{-g} switch tells the compile to include debugging information in the executable file.
  This increases the exe file size but helps programs like \texttt{GDB} understand what is going on.

\item
  Test that the program runs as normal using:
\begin{verbatim}
./program_monolith.exe
\end{verbatim}

\item
  Now we are going to run it using \texttt{gdb}:
\begin{verbatim}
gdb program_monolith.exe
\end{verbatim}

\item
  GDB is a command-line program but has basic full-screen capabilities to show code, assembly language and more.
  To enable the full-screen mode we type:
\begin{verbatim}
tui enable
\end{verbatim}

\item
  Then we can split the screen between the C code and its assembler.
\begin{verbatim}
layout split
\end{verbatim}

\item
  To keep a record of your session we will turn logging on:
\begin{verbatim}
set logging enabled
\end{verbatim}

\item
  If we just ran the program now it would run same as before.
  We want to get it to pause just after starting.
  To do that we set a \textbf{breakpoint} at the beginning of the \texttt{main} function.
\begin{verbatim}
break main
\end{verbatim}
  You should see a \texttt{b+} appear in the code and disassembly.

\item
  Start the program by typing
\begin{verbatim}
run
\end{verbatim}
You should now see the first line highlighted in both the C and assembly. 

\item
  Use the \texttt{next} command repeatedly to get to the line that sets \texttt{x}.

\item \label{step:gdb-memory-lab:p-pc-sp}
  Use the \texttt{p \$pc} and \texttt{p \$sp} commands to show the program counter and stack pointer values.

\item
  Use next to step to to the next instruction once.

\item
  Using the same commands from step~\ref{step:gdb-memory-lab:p-pc-sp}, which one has changed and do you know why?

\item
  Using the command \texttt{x/10x \$sp} to print the first 10 items in the stack can you identify the variable \texttt{x} on the stack?

\item
  Use \texttt{finish} to program finish executing.  

\item
  Repeat the same analysis with the \texttt{program\_modular.c} example. 
  
\end{enumerate}


\section{Submission}

Use git to add all the C files and the gdb.txt log file in today's lab to your repository.
Then Commit and push before leaving.

