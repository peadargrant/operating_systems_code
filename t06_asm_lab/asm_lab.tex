\chapter{Assembly language lab}
\label{ch:assembly-language-lab}

In this lab we will do some basic exercises involving assembly language.
The aim of this lab are:
\begin{enumerate}
\item To let you practically see the ideas from our discussion on programs.
\item To see how assembly language differs among different processor families.
\item To build some familiarity with the UNIX tools at the command-line.
\end{enumerate}

\section{Pre-requisites}

Before doing the rest of this lab make sure that you have:
\begin{enumerate}
\item Set up your GitLab repository for lab work. (\autoref{ch:gitlab-setup})
\item Created a linux VM on XOA. (\autoref{ch:xoa-setup})
\item Cloned your gitlab repository onto your VM into \texttt{os\_labs}.
\end{enumerate}

\section{Exercise}

All of these steps are to be completed on your XOA VM.

\begin{enumerate}

\item
  Use \texttt{cd} to change into the \texttt{os\_labs} folder.

\item
  Make a new folder called \texttt{week03}.

\item
  Change directory into your \texttt{week03} folder.
  (To check you're in the right place use \texttt{pwd} command.)
  
\item 
  Use the \texttt{nano} text editor to create the following C file named \texttt{hello.c}:\\
  \inputminted{c}{hello.c}

\item
  Compile this C program to an executable program using the command:
\begin{verbatim}
gcc -o hello hello.c
\end{verbatim}
  The \texttt{-o} option tells \texttt{gcc} to name the output file \texttt{hello}.

\item
  Run this program by typing:
\begin{verbatim}
./hello
\end{verbatim}
  and confirm that it does in fact print ``Hello world''.

\item \label{step:asm-creation}
  The \texttt{gcc} compiler by default takes C code and produces machine code.
  We can get it instead to output assembly language using the \texttt{-S switch}.
  Try it to get \texttt{hello.asm} as follows:
\begin{verbatim}
gcc -s -o hello.asm hello.c 
\end{verbatim}

\item \label{step:add-to-git}
  Add \texttt{hello.c} to your git repository:
\begin{verbatim}
git add hello.c
\end{verbatim}

\item
  Open the \texttt{hello.asm} file using \texttt{nano}.
  It's not particularly understandable is it?

\item
  Add \texttt{hello.asm} to git using the same command as in step \ref{step:add-to-git}.
  
\item
  Let's now make a simpler file named \texttt{add2sub1.c}.
  This is a function, again in C, that adds two numbers and subtracts a third:
  \inputminted{c}{add2sub1.c}

\item
  Use \texttt{gcc} to convert this to \texttt{add2sub1.asm} (using step \ref{step:asm-creation} as a guide).

\item
  Open \texttt{add2sub1.asm} in \texttt{nano}.
  Can you find the add and subtract instruction in assembly language?

\item
  Add \texttt{add2sub1.c} and \texttt{add2sub1.asm} to git.
  
\item
  Now let's see what happens if we try compiling to assembly language for an ARM processor instead.
  To do this we'll install a cross compiler for ARM on our \texttt{x86\_64} virtual machine/ 
  Use \texttt{apt} to install the following packages:\\
  \texttt{gcc-aarch64-linux-gnu} and \texttt{binutils-aarch64-linux-gnu}
  
\item 
  Create \texttt{add2sub1\_aarch64.asm} using the command:
\begin{verbatim}
aarch64-linux-gnu-gcc add2sub1.c -s -o add2sub1_aarch64.asm
\end{verbatim}

\item
  Examine the \texttt{add2sub1\_aarch64.asm} file in a text editor.
  Does this look different to the \texttt{x86\_64} assembly language?
  Can you find the add and subtract instructions?

\item
  Add \texttt{add2sub1\_aarch64.asm} to git.
  
\end{enumerate}

\section{Recording your lab work}

\begin{enumerate}
\item Type
\begin{verbatim}
git status
\end{verbatim}
  and confirm that all the files required above have been added to git.
\item
  Commit your work to your Git repository on your XOA VM with a descriptive message.
\begin{verbatim}
git commit -m 'week 3 lab work on assembly language'
\end{verbatim}
\item
  Push your commits to GitLab using
\begin{verbatim}
git push
\end{verbatim}
\end{enumerate}

Your work is done for today!

