\chapter{Free space management}
\label{ch:free-space-management}

% \begin{goals}
%   \begin{enumerate}
%   \item \textbf{Identify} the challenges of free space management.
%   \item \textbf{Identify} the different types of fragmentation that may occur.
%   \item \textbf{Describe} the mechanisms for managing free space.
%   \item \textbf{Explain} principal allocation algorithms.
%   \end{enumerate}
% \end{goals}

% \begin{reading}
%   \begin{enumerate}
%   \item
%     \citet[Chapter 17]{arpaci-dusseau:2015:operating}\\
%     \url{http://pages.cs.wisc.edu/~remzi/OSTEP/vm-freespace.pdf}
%   \end{enumerate}
% \end{reading}


\section{Fragmentation}

Once variable-size segments are permitted, the issue of fragmentation of available free space arises.

\begin{description}
\item[External] fragmentation is when available free space is divided among units of different sizes.
\item[Internal] fragmentation is when an allocator gives out more memory than asked for.
\end{description}

Both internal and external fragmentation involve unusable wasted free space.

\section{Broad approaches}

\begin{description}
\item[Compaction] is where the OS attempts to reorganise memory to remove fragmentation.
  Bad for performance since copying operations take a lot of time.
\item[Free list] is where the OS maintains a list of free memory locations.
  When memory is requested, it chooses using a number of possible algorithms.
\end{description}


\section{Assumptions}

\begin{enumerate}
\item Basic allocation (\texttt{malloc / new}) and deallocation (\texttt{free / delete}) interfaces.
\item Dealing with external fragmentation only.
\item Memory cannot be relocated once allocated.
\end{enumerate}


\section{Mechanisms}

\subsection{Splitting}

A contiguous region of memory is divided into two (possibly unequally sized) regions of memory, of which one is usually allocated.

\subsection{Coalescing}

Two contiguous regions of memory may exist that are listed separately as free.
Coalescing joins the contiguous free regions to form one larger one.


\section{Tracking free space}

Tracking of free space can be done easily by including a \textbf{header} before the beginning of each allocated memory block.
The header has the size of the allocated block.

Free space also has a header, that specifies the size of the free space and also the address of the next chunk of free space.

\section{Allocation algorithms}

\subsection{Best fit}
\label{sec:best-fit}

Gives location closest in size to that requested:
\begin{itemize}
\item Simple theoretical implementation: reduces wasted space.
\item Performance penalty due to search.
\end{itemize}


\subsection{Worst-fit}
\label{sec:worst-fit}

Finds the largest chunk and splits it so as to allocate the requested size to the process:
\begin{itemize}
\item Attempts to keep large chunks of memory free.
\item Similar performance penalty to best fit.
\item Shown to have excessive fragmentation.
\end{itemize}


\subsection{First fit}
\label{sec:first-fit}
 
Assigns the first block of memory large enough for the request:
\begin{itemize}
\item Faster than best fit and worst fit.
\end{itemize}

\subsection{Next fit}
\label{sec:next-fit}

Same as first fit, but the search picks up from the previous location on subsequent allocation requests.
\begin{itemize}
\item Spreads searches throughout free space more uniformly.
\end{itemize}

\subsection{Buddy}

Splits free space until request exceeds half of free block.


\section{Remaining issues}

Segmentation allowed the physical memory space to be divided and allocated in chunks of varying size.
The resultant fragmentation complicates allocation.
We'll next expore a different approach.


