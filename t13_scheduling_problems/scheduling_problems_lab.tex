\chapter{Scheduling problems}
\label{ch:scheduling-problems}

Complete these exercises on paper.
Then type the solution as best you can into a text file named \texttt{scheduling\_exercises.txt} in the \texttt{week04} folder of your \texttt{os\_labs} repository.

\begin{enumerate}

\item 
  A job arrives into a system at t=0, starts executing at t=3 and completes at t=8.
  \begin{enumerate}
  \item What is the processing time for this job?
  \item What is the turnaround time for this job?
  \item What is the response time for this job?
  \end{enumerate}

\item \label{question:scheduling-lab:fifo-jobs}
  4 jobs arrive into a FIFO scheduler at approximately the same time (t=0) in order:\\
  Job 0 takes 300 seconds to complete, Job 1 takes 40 seconds, Job 2 takes 30 seconds and Job 4 takes 50 seconds.
  \begin{enumerate}
  \item What order will the jobs run in?
  \item Calculate the turnaround time for Job 2.
  \item Calculate the average turnaround time for the system.
  \end{enumerate}
  
\item
  The same set of jobs as in question~\ref{question:scheduling-lab:fifo-jobs} arrive into an SJF scheduler.
  \begin{enumerate}
  \item What order wil the jobs run in?
  \item Calculate the turnaround time for Job 2.
  \item Calculate the average turnaround time for the system.
  \item What is the percentage change of the average turnaround time compared to the FIFO scheduler?
  \item What phenomenon has switching from FIFO to SJF avoided?
  \end{enumerate}

\item \label{question:scheduling-lab:fablab-jobs}
  A local digital fabrication lab implements a scheduling system to control a CNC mill.
  Jobs are submitted by FabLab users directly from CAD workstations.
  Jobs arrive as follows:
\begin{verbatim}
Job  Arrives     Run-time
===  ==========  =========
  0    09:00:02   10 mins
  1    09:00:03    2 hours
  2    09:00:05    2 hours
  3    09:04:35    1 hour
  4    09:32:36   30 mins
  5    10:24:01   20 mins
\end{verbatim}
  Assume that the scheduler update the queue every minute on the minute.
  The scheduler is placed in FIFO mode.
  From this information determine:
  \begin{enumerate}
  \item The order that the jobs will run in.
  \item The turnaround time for Job 2.
  \item The average turnaround time.
  \item The average response time.
  \end{enumerate}

\item Utilising the same information as in question~\ref{question:scheduling-lab:fablab-jobs}, the scheduler is instead operated in SJF mode.
  Determine:
  \begin{enumerate}
  \item The order that the jobs will run in.
  \item The turnaround time for Job 2.
  \item The average turnaround time.
  \item The average response time.
  \end{enumerate}

\end{enumerate}

Add, commit and push the solution to GitLab.
