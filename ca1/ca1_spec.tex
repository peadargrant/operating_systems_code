\documentclass[10pt]{article}

\title{CA1 specification}
\author{Dr Peadar Grant}
\date{\today}

\usepackage{mathptmx}
\usepackage[margin=1.5cm,bmargin=2.5cm]{geometry}
\usepackage{hyperref}

\begin{document}

\maketitle

\begin{description}
\item[Due date:] as displayed on Moodle.
\item[Contribution to Module Mark:] as on module descriptor.
\end{description}

\section{Grading}

Grading will be applied using the following categories for each criterion in the specification:
\begin{center}
  \textbf{No grade} (N=0) --- \textbf{Unsatisfactory} (U=20) -- \textbf{Poor} (P=40) -- \textbf{Good} (G=60) -- \textbf{Excellent} (E=80) -- \textbf{Exceptional} (X=100)
\end{center}

\section{Specification}

You are to build a simulation tool that demonstrates a number of different job scheduling algorithms that we have discussed in class.

\subsection{Scheduling simulator (55\%)}

Your program must obey the following calling convention:
\begin{verbatim}
./simulate.py jobs.json
\end{verbatim}
where \texttt{jobs.json} is the input JSON file.

\subsubsection{Input}

Input is to be a JSON file in the format:
\begin{verbatim}
[
  {
    "name": "J0",
    "arrival": 76,
    "duration": 5
  },
  {
    "name": "J1",
    "arrival": 59,
    "duration": 0
  },
  {
    "name": "J2",
    "arrival": 44,
    "duration": 1
  }
]
\end{verbatim}
You can assume that the runtime is known absolutely and that no I/O or other blocking operations take place.

\subsubsection{Output}

It must show at each time step the process being run using FIFO, SJF, STCF, RR scheduling (with two different time slice values) as follows:

\begin{verbatim}
T  FIFO SJF  RR1 RR2
1  JOB1 JOB2 UNKNOWN UNKNOWN
2  JOB1 JOB1 JOB1 JOB 2
\end{verbatim}

when any of the following happen, it should print the necessary notifications on a line of their own before the relevant time step: 

\begin{verbatim}
* ARRIVED: MY_JOB
* COMPLETE: JOB1
\end{verbatim}

The scheduler should continue simulating until no jobs run, printing:

\begin{verbatim}
= SIMULATION COMPLETE, NO JOBS LEFT
\end{verbatim}

The mechanism you use is entirely up to you. You can either pre-compute the values for each scheduling algorithm or you can run each at each time step - there are different trade-offs inherent in each approach.


\subsection{Per-job statistics (15\%)}

Once simulation is complete, your program must output per-job statistics for turnaround time and response time in the following format: 

\begin{verbatim}
= INDIVIDUAL PER-JOB STATISTICS
# JOB FIFO SJF STCF RR1 RR2
T JOB1 10.1 5.2 21.2 4.0 5.0
T JOB2 10.4 3.3 12.3 13.4 12.0

\end{verbatim}

You must then print the response times in a similar fashion with \texttt{R} replacing \texttt{T}.


\subsection{Aggregate statistics (15\%)}

Your program must then output the average turnaround and response times for each scheduling algorithm:

\begin{verbatim}
= AGGREGATE STATS COMPLETE
# SCHEDULER AVG_TURNAROUND AVG_RESPONSE
@ SJF 10.53 30.2 
@ FIFO 32.2 19.1
... (put in other schedulers)
@ RR2 40.0 10.0
\end{verbatim}

\subsection{Demo scripts (15\%)}

Supply a demo script to run your program 3 times.
Each demo must be able to be invoked as \texttt{./demo.sh}

Each of the three demo cases should generate files with different overall arrival time and runtime ranges using the \url{ca1_jobs.py} script.
Run the script with no arguments to see the required command-line parameters.
There must be at least 10 jobs with a max runtime of at least 5 in all 3 demo runs.

Your program should print the program output and save it also to \texttt{demo\_output1,2,3.txt} etc as appropriate.

\subsection{Source control (10\%)}

You are required to maintain your work in your \texttt{os\_labs} git repository under a folder \texttt{ca1}. 
It is required that:

\begin{itemize}
  \item commits are atomic
  \item clear development timeline evident from history,
  \item meaningful concise commit messages,
\end{itemize}

\section{Requirements for grading}

Please note the following requirements carefully:

\begin{itemize}
\item Files must be presented in Git repo.
\item All files must have the correct names including case.
\end{itemize}

If \textit{any} of the above requirements are not satisfied, the file will be treated as not existing, and you \textbf{work will receive zero}. Depending on what file is missing, this may cause other components to fail and receive zero.


\section{Academic integrity}

You are reminded that the DkIT academic integrity policy will apply in full.
Any suspected breaches of academic integrity will be progressed through the official processes without exception.
\textit{If you are unsure if a particular course of action is acceptable you must check prior to submission!}

\end{document}
