\chapter{Scheduling 2}
\label{ch:scheduling-2}

% \begin{goals}
%   \begin{enumerate}
%   \item \textbf{Quantify} scheduling using response time.
%   \item \textbf{Analyse} time slicing using round robin scheduling.
%   \end{enumerate}
% \end{goals}

% \begin{reading}
%   \begin{enumerate}
%     \item
%       \citet[Chapter 5]{arpaci-dusseau:2015:operating}\\
%       \url{http://pages.cs.wisc.edu/~remzi/OSTEP/cpu-sched.pdf}
%     \end{enumerate}
% \end{reading}

\section{Metrics}

The use of the turnaround time is most appropriate when describing non-interative batch processes that do not give any feedback when running.
Alternatively, we may be interested in the response time: 

\subsection{Response time}
\label{sec:response-time}

The response time, $T_R$, is defined as: 
\begin{align}
  T_R = T_F - T_A \label{eq:response-time}
\end{align}
where $T_F$ is the first-run time of the process (i.e. when it first starts) and $T_A$ denotes the arrival time as before.

\section{Round-Robin (RR) scheduling}
\label{sec:rr}

Round-Robin (RR) scheduling, also called time slicing, runs each process for a predefined slice of time.

\subsection{Time slice length}

Time slice length is the key parameter for RR scheduling:
\begin{itemize}
\item Make it \textbf{short enough} to attain required response time.
\item Make it \textbf{long enough} to amortize the cost of process switching.
\end{itemize}

\subsection{Costs of context switching}

\begin{itemize}
\item Saving and restoring machine registers is low-cost, but:
\item Modern chips have a lot of hardware-based performance boosts: caches, translation lookaside buffers (see later), branch predictors etc.
\end{itemize}

\subsection{Suitability}

RR is technically a fair scheduling algorithm (response time), but at the cost of performance (turnaround time).

\section{Incorporating I/O}

Two assumptions remain:
\begin{itemize}
\item each process does no I/O (\ref{item:scheduling-assumption:cpu-only} from \autoref{sec:scheduling-assumptions})
\item that the runtime is known a priori (\ref{item:scheduling-assumption:run-time-known} from \autoref{sec:scheduling-assumptions})
\end{itemize}

\subsection{I/O}

An I/O request \textit{blocks} a process until it completes:
\begin{itemize}
\item A scheduler that waits for the I/O operation to complete is wasting CPU time.
\item Can segment a job into multiple CPU burst sub-jobs.
\item Schedule other jobs (or sub-jobs) while waiting for blocking operation to complete.
\end{itemize}




%%% Local Variables:
%%% mode: latex
%%% TeX-master: "os_notes"
%%% End:
