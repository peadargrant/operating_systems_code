\chapter{Page table size}
\label{ch:page-table-size}

% \begin{goals}
%   \begin{enumerate}
%   \item \textbf{Identify} problems with linear per-process page tables.
%   \item \textbf{Propose} solutions for reducing memory requirements of page tables.
%   \item \textbf{Evaluate} increased page size and hybrid paging/segmentation models.
%   \end{enumerate}
% \end{goals}

% \begin{reading}
%   \begin{enumerate}
%   \item
%     \citet[Chapter 20]{arpaci-dusseau:2015:operating}\\
%     \url{http://pages.cs.wisc.edu/~remzi/OSTEP/vm-smalltables.pdf}
%   \end{enumerate}
% \end{reading}

\section{Large page tables}

\begin{itemize}
\item Every process in a system has its own page table.
\item Every page table must hold validity (true/false) for all virtual addresses, and if valid map each to a physical address.
\item Large overhead of memory required for all page tables, want to reduce this!
\item Also, consider a situation where the utilised portions of the code, heap and stack space are small within the virtual address space.
\end{itemize}

\section{Increased page size}

Consider an 8-bit address space, with the pattern:
\begin{verbatim}
VPN     OFFSET
======= =======
V V V V O O O O
\end{verbatim}
With this arrangement, the VPN is 4-bits in size and requires $2^4=16$ entries.
Consider what would happen if we increased the page table size.
If we made the offset 5-bits, and consequently a 3-bit VPN:
\begin{verbatim}
VPN   OFFSET
===== =========
V V V O O O O O
\end{verbatim}
The page table itself now requires only $2^3=8$ entries.

\subsection{Problem}

Larger pages waste space, known as internal fragmentation.

As page sizes get larger, proportionally less of the allocated space will actually be used.

\section{Hybrid paging and segmentation}

Hybrid segmentation/paging primarily addresses the sparce address space problem.
Dividing the classical virtual address space model into pages ignores its common usage pattern.

Idea: have a page table per logical segment.
Recall:
\begin{description}
\item[Base] register holds physical address of segment.
\item[Bounds] register shows segment size.
\end{description}
Use the MMU-resident base/bounds register to hold start/end addresses of page table for that segment, so here:
\begin{description}
\item[Base] register holds physical address of page table for segment.
\item[Bounds] register shows page table size (i.e. how many pages).
\end{description}

\subsection{Example}

Imagine an 16-bit address space, with the following arrangement:
\begin{itemize}
\item Upper 2 bits denote segment. (Explicit)
\item Next 6 bits denote page.
\item Last 8 bits denote offset.
\end{itemize}
as in:
\begin{verbatim}
S S V V V V V V O O O O O O O O
\end{verbatim}

\subsection{TLB miss handling}

\begin{enumerate}
\item Segment \texttt{S} bits determine segment, base+bounds pair.
\item VPN got by shifting VPN bits.
\end{enumerate}

\subsection{Problems}

Key problems prevent this idea being used more often: 

\begin{itemize}
\item Assumes a strict adherence to the code/heap/stack pattern.
\item External fragmentation can arise again, now have page tables of different sizes.
\end{itemize}


